\section{运动和力的关系}

\subsection{牛顿第一定律}
\Itemize{
\item \Keyword{牛顿第一定律}{Newton's first law},即一切物体总保持匀速直线运动状态或静止状态,除非作用在它上面的力迫使它改变这种状态。
\item 运动状态改变,即一个物体由静止变为运动或由运动变为静止,或一个物体的速度大小或方向改变了。
\item 牛顿第一定律所描述的状态是一种理想状态,不可能用实验直接验证。
\item 牛顿第一定律的意义,包括\blue{发现了惯性},\blue{揭示了运动和力的关系并定义了力},\blue{定义了惯性参考系}。
\item 物体保持原来匀速直线运动状态或静止状态的性质叫做\Keyword{惯性}{inertia}。牛顿第一定律也叫做\blue{惯性定律}。
\item 惯性是物体的\blue{固有属性}。一切物体都具有惯性。
\item 力不是维持物体运动状态的原因,而是改变物体运动状态的原因。
\item 牛顿第一定律是否成立与参考系的选择有关。
\item 牛顿第一定律成立的参考系叫做\blue{惯性参考系},简称\blue{惯性系};牛顿第一定律不成立的参考系叫做\blue{非惯性参考系},简称\blue{非惯性系}。
\item \blue{所有惯性系对一切物理规律都是等价的}。
}

\subsection{实验:探究加速度与力、质量的关系}
\Itemize{
\item \blue{阻力补偿法}。
}

\subsection{牛顿第二定律}
\Itemize{
\item \Keyword{牛顿第二定律}{Newton's second law},即物体加速度的大小跟它受到的作用力成正比,跟它的质量成反比,加速度的方向跟作用力的方向相同,即:
$$a\propto\frac Fm~~或~~F=kma$$
其中 $F$ 是物体所受的\blue{合力}。
\item \blue{规定}使质量为 1kg 的物体获得 $\text{1m/s}^2$ 的加速度的力为 $\text{1kg}\cdot\text{m/s}^2$,并叫做\blue{牛顿},符号是 \blue{$\bf N$},即 \blue{$\bf1N=1kg\cdot m/s^2$},此时 $k=1$。即:
\Math{F=ma}
\item 牛顿第二定律的意义,包括\blue{定义了质量},\blue{进一步揭示了运动和力的关系}。
\item \Keyword{质量}{mass}是描述物体惯性大小(物体运动状态变化难易程度)的物理量,也叫做\blue{惯性质量}。
\item 在国际单位制中,质量的单位是\blue{千克},符号为 \blue{$\bf kg$}。
\item 质量大的物体惯性大。质量越大,物体的运动状态越难改变。
\item 加速度是连接运动学和动力学的桥梁。
}

\subsection{牛顿运动定律的应用}
\Itemize{
\item \blue{从受力确定运动情况}和\blue{从运动情况确定受力}。
\item \blue{求加速度是关键}。
}

\subsection{超重和失重}
\Itemize{
\item 体重计的示数称为\blue{视重},反映了人对体重计的压力。
\item 物体对支持物的压力(或对悬挂物的拉力)大于物体所受重力的现象叫做\Keyword{超重}{overweight}。此时,\blue{视重大于实重},即 $F_{\text{N}}=m(g+a)>mg$。
\item 物体对支持物的压力(或对悬挂物的拉力)小于物体所受重力的现象叫做\Keyword{失重}{weightlessness}。此时,\blue{视重小于实重},即 $F_{\text{N}}=m(g-a)<mg$。
\item 物体对支持物(或悬挂物)完全没有作用力的现象叫做\blue{完全失重}。此时,\blue{视重等于零},即 $F_{\text{N}}=0$。
}

\subsection{物理量与单位制}
\Itemize{
\item 物理量,应具有明确的定义和物理意义,可以用不同的方法测量,测量的结果用数值和相应的单位来表示。
\item 物理学的关系式在确定物理量的数值之间的关系时,也确定了物理量之间的关系。
\item 被选定的相互独立的物理量叫\blue{基本物理量},简称\blue{基本量},相应的单位叫做\blue{基本单位}。
\item 由基本量根据物理关系推导出来的其他物理量叫做\blue{导出物理量},简称\blue{导出量},相应的单位叫做\blue{导出单位}。
\item 基本单位和导出单位一起组成一个\Keyword{单位制}{systems of units}。
\item 1960 年第 11 届国际计量大会制定了一种国际通用的、包括一切计量领域的单位制,叫做\Keyword{国际单位制}{Le Système Internation d'Unités},简称 \blue{SI}。
\item 力学中的基本量包括\blue{时间}、\blue{长度}、\blue{质量},基本单位包括\blue{秒}、\blue{米}、\blue{千克}。
}