\section{磁}

\subsection{磁现象}
\Itemize{
\item 能够吸引\blue{铁}、\blue{钴}、\blue{镍}等物质的性质叫做\Keyword{磁性}{magnestion}。
\item 具有磁性的物体叫做\Keyword{磁体}{magnet}。
\item 磁性最强的两个部位叫做\Keyword{磁极}{magnetic pole}。
\item 能够自由转动的磁体,静止时由指向北方的磁极叫做\Keyword{北极}{north pole}或 \blue{N 极},指向南方的磁极叫做\Keyword{南极}{south pole}或 \blue{S 极}。
\item 磁极间相互作用的规律,即\blue{同名磁极相互排斥},\blue{异名磁极相互吸引}。
\item 原本没有磁性的物体在磁体或电流的作用下获得磁性的过程叫做\Keyword{磁化}{magnestization}。
\item 能够被磁化的物质统称为\blue{磁性材料}。
\item 磁性材料分为硬磁性材料和软磁性材料。
\item 被磁化后能够长期保持磁性的材料叫做\blue{硬磁性材料}(永磁体)。被磁化后不能长期保持磁性的材料叫做\blue{软磁性材料}。
}

\subsection{磁场}
\Itemize{
\item \Keyword{磁场}{magnestion field}的\blue{基本性质},即磁场对放入其中的磁体有力的作用。
\item 磁场中的不同位置磁场的强弱和方向不同。
\item \blue{规定}小磁针静止时 \blue{N 极所指的方向}为这点磁场的方向。
\item \Keyword{磁感线}{magnetic induction line}\blue{不存在},只是为了方便形象地描述磁场。
\item 磁体外部的磁感线都是从磁体的 \blue{N 极出发回到 S 极}。
\item \blue{磁感线疏密表示磁场强弱}。磁感线稀疏的地方磁场弱,磁感线密集的地方磁场强。
\item 磁感线上某点的\blue{切线方向},既是放在该处的小磁针 \blue{N 极的受力方向},也是该点的\blue{磁场方向}。
\item 地球周围空间存在的磁场叫做\blue{地磁场}。
\item 地磁的 N 极在地理的南极附近,地磁的 S 极在地理的北极附近。
\item 地磁场的两级和地理的两级不重合。
\item 地磁场的磁感线分布跟条形磁体的磁场相似。
\item 磁针所指南北方向偏离地理南北方向的角度叫做\blue{磁偏角}。
}

\subsection{电流的磁感应}
\Itemize{
\item 奥斯特实验。
\item 通电导线周围存在与电流方向有关的磁场,这种现象叫做\blue{电流的磁效应}。
\item 通电\blue{螺线圈}(\blue{线圈})外部的磁场与条形磁体的磁场相似。
\item \Keyword{安培定则}{Anpere's rule}或\Keyword{右手螺旋定则}{right-handed screw rule}。
}

\subsection{电磁铁及其应用}
\Itemize{
\item \Keyword{电磁铁}{electromagnet},即\blue{线圈}与\blue{铁芯}的组合。
\item 有电流时产生磁性,没有电流时失去磁性。
\item 匝数一定时,\blue{电流}越大,电磁铁的磁性越强。电流一定时,\blue{匝数}越多,电磁铁的磁性越强。
\item \blue{继电器}是利用低电压、弱电流电路的通断,来间接地控制高电压、强电流电路通断的装置。
\item \blue{电磁继电器}是利用电磁铁来控制工作电路的一种开关。其工作电路由\blue{低压控制电路}和\blue{高压工作电路}两部分构成。
}

\subsection{安培力与电动机}
\Itemize{
\item 通电导线在磁场中受到力的作用,力的方向跟电流的方向、磁场的方向有关,这个力叫做\blue{安培力}。
\item \Keyword{左手定则}{left-hand rule}。
\item 电动机是将电能转化为其他形式能的装置。
\item 电动机分为\blue{直流电动机}和\blue{交流电动机}。
\item 电动机由转子和定子两部分组成。能够转到的部分(线圈)叫做\blue{转子},固定不动的部分(磁体)叫做\blue{定子}。
}

\subsection{电磁感应与发电机}
\Itemize{
\item \blue{闭合电路}的\blue{一部分导体}在磁场中做\blue{切割磁感线}运动时,导体中就产生电流,这种现象叫做\Keyword{电磁感应}{electromagnetic induction}。产生的电流叫做\Keyword{感应电流}{induction current}。
\item \Keyword{右手定则}{right hand rule}。
\item 发电机是将其他形式能转化为电能的装置。
\item 发电机分为\blue{直流发电机}和\blue{交流发电机}。
\item 发电机由\blue{转子}(转动部分)和\blue{定子}(固定部分)两部分组成。
\item 方向随时间变化的电流叫做\Keyword{交变电流}{alternating current},简称\blue{交流},符号 \blue{AC}。方向不随时间变化的电流叫做\Keyword{直流电流}{direct current},简称\blue{直流},符号 \blue{DC}。
\item 交变电流的频率在数值上等于电流在每秒内周期性变化的次数。
\item 我国电网以交流供电,频率为 \blue{50Hz}。
}