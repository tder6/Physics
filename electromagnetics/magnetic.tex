\section{磁}

\subsection{磁现象}
\begin{itemize}
\item 能够吸引\blue{铁}、\blue{钴}、\blue{镍}等物质的性质叫做\blue{磁性}(magnestion).
\item 具有磁性的物体叫做\blue{磁体}(magnet).
\item 磁性最强的两个部位叫做\blue{磁极}(magnetic pole).
\item 能够自由转动的磁体,静止时由指向北方的磁极叫做\blue{北极}(north pole)或 \blue{N 极},指向南方的磁极叫做\blue{南极}(south pole)或 \blue{S 极}.
\item 磁极间相互作用的规律,即\blue{同名磁极相互排斥},\blue{异名磁极相互吸引}.
\item 原本没有磁性的物体在磁体或电流的作用下获得磁性的过程叫做\blue{磁化}(magnestization).
\item 能够被磁化的物质统称为\blue{磁性材料}.
\item 磁性材料分为硬磁性材料和软磁性材料.
\item 被磁化后能够长期保持磁性的材料叫做\blue{硬磁性材料}(永磁体). 被磁化后不能长期保持磁性的材料叫做\blue{软磁性材料}.
\end{itemize}