\section{电荷}

\vspace{10pt}
\Itemize{
\item 物体能够吸引轻小物体,就说物体带了电,即物体带了\Keyword{电荷}{electric charge}。带了电荷的物体叫做\blue{带电体}。
\item 使物体带电叫做\blue{起电}。用摩擦的方式使物体带电叫做\Keyword{摩擦起电}{electrification by friction}。
\item 自然界\blue{只有}两种电荷。
\item 用丝绸摩擦过的玻璃棒带的电荷叫做\Keyword{正电荷}{positive charge}。用毛皮摩擦过的橡胶棒带的电荷叫做\Keyword{负电荷}{negative charge}。
\item \blue{同种}电荷相互\blue{排斥},\blue{异种}电荷相互\blue{吸引}。
\item 电荷的多少叫做\Keyword{电荷量}{electric quantity},简称\blue{电量}。用 \blue{$\bm Q$} 或 \blue{$\bm q$} 表示。在国际单位制中,电荷量的单位是\Keyword{库仑}{coulomb},简称\blue{库}。符号是 \blue{$\bf C$}。正电荷的电荷量为正值,负电荷的电荷量为负值。
\item \blue{验电器}和\blue{静电计}。
\item 两种电荷互相完全抵消叫做\blue{中和}。
\item 物质是由\blue{分子}构成的,分子是由\blue{原子}构成的。
\item 原子是由带正电的\blue{原子核}和带负电的\Keyword{电子}{electron}组成的。
\item 原子核是由带正电的\blue{质子}和不带电的\blue{中子}组成的。
\item 每个原子中质子与电子的\blue{数量相等},质子与电子所带的\blue{电荷量相同}。
\item 摩擦起电的本质是电荷从一个物体\blue{转移}到另一个物体。
\item 金属原子中能脱离原子核的束缚而在金属中自由运动的电子叫做\Keyword{自由电子}{free electron}。
\item 失去自由电子的原子叫做\Keyword{离子}{ion}。
\item 质子、电子所带的电荷量(\blue{最小的电荷量})叫做\Keyword{元电荷}{elementary charge},用 \blue{$\bm e$} 表示。有:
\Math{e\approx1.6\times10^{-19}\text{C}}
\item 所有带电体的电荷量都是 $e$ 的整数倍,不是连续变化的,即量子化的。
\item 电子的电荷量 $e$ 与质量 $m_e$ 之比叫做\Keyword{电子的比荷}{specific charge}。电子的质量 $m_e=9.11\times10^{-31}\text{kg}$,则电子的比荷为:
$$
\frac{e}{m_e}\approx1.76\times10^{11}\text{C}/\text{kg}
$$
\item 利用静电感应使金属带电叫做\Keyword{感应起电}{electrification by induction},所带电荷叫做\Keyword{感应电荷}{induced charge}。
\item \Keyword{静电感应}{electrostatic induction}。
\item 三种常见的起电方式包括摩擦、接触、感应。
\item \Keyword{电荷守恒定律}{law of conservation of charge},即电荷既不会创生,也不会消灭。它只能从一个物体转移到另一个物体,或者从物体的一部分转移到另一部分。在转移的过程中,电荷的总量保持不变。
\item 一个与外界没有电荷交换的系统,电荷的\blue{代数和}保持不变。
\item 电荷守恒定律是自然界最普遍、最重要的基本定律之一。
}