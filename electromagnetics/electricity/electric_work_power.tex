\section{电功和电功率}

\subsection{电功和电能}
\Itemize{
\item \Keyword{电能}{electric energy}可以转化为其他形式的能。单位是\blue{焦耳},简称\blue{焦},符号是 \blue{$\bf J$}。常用单位还有\blue{千瓦时},简称\blue{度},符号是 \blue{$\bf kW\cdot h$}。换算关系是 \blue{$\bf 1kW\cdot h=3。6\times10^6J$}。
\item 电流做的功叫做\Keyword{电功}{electric work}。用 \blue{$\bm W$} 表示。单位是\blue{焦耳},简称\blue{焦},符号是 \blue{$\bf J$}。
\item 电流做了多少功,就有多少电能转化为其他形式的能。
\item 电功等于电压 $U$、电流 $I$ 和通电时间 $t$ 的乘积,即:
\Math{W=UIt}
\item 根据 $U=IR$,电流通过电阻 $R$ 做的功为:
$$
W=I^2Rt~~或~~W=\frac{U^2}Rt
$$
\item 电功或电能的计量仪器叫做\blue{电能表}(电度表)。
}

\subsection{电功率}
\Itemize{
\item \Keyword{电功率}{electric power}是表示\blue{电流做功快慢}的物理量。用 \blue{$\bm P$} 表示。单位是\blue{瓦特},简称\blue{瓦},符号是 \blue{$\bf W$}。常用单位还有千瓦(\blue{$\bf kW$})、毫瓦(\blue{$\bf mW$})。换算关系是 \blue{$\bf1kW=10^3W$},\blue{$\bf1mW=10^{-3}W$}。
\item 电功率等于电流 $U$ 和电压 $I$ 的乘积,即:
\Math{P=\frac Wt=UI}
\item 根据 $U=IR$,电流通过电阻 $R$ 的电功率为:
$$
P=I^2R~~或~~P=\frac{U^2}R
$$
\item 用电器正常工作时的电压叫做\Keyword{额定电压}{rated voltage},用电器在额定电压下工作时的电功率叫做\Keyword{额定功率}{rated power}。
}

\subsection{焦耳定律}
\Itemize{
\item 电流通过导体的电能转化为内能,这种现象叫做\blue{电流的热效应}。
\item 电流的效应包括热效应、磁效应和化学效应。
\item \Keyword{焦耳定律}{Joule's law},即电流通过导体产生的热量 $Q$ 跟电流 $I$ 的二次方成正比,跟导体的电阻 $R$ 成正比,跟通电时间 $t$ 成正比。即:
\Math{Q=I^2Rt}
}