\section{电磁感应与发电机}

\Itemize{
\item \blue{闭合电路}的\blue{一部分导体}在磁场中做\blue{切割磁感线}运动时,导体中就产生电流,这种现象叫做\Keyword{电磁感应}{electromagnetic induction}。产生的电流叫做\Keyword{感应电流}{induction current}。
\item \Keyword{右手定则}{right hand rule}。
\item 发电机是将其他形式能转化为电能的装置。
\item 发电机分为\blue{直流发电机}和\blue{交流发电机}。
\item 发电机由\blue{转子}(转动部分)和\blue{定子}(固定部分)两部分组成。
\item 方向随时间变化的电流叫做\Keyword{交变电流}{alternating current},简称\blue{交流},符号 \blue{AC}。方向不随时间变化的电流叫做\Keyword{直流电流}{direct current},简称\blue{直流},符号 \blue{DC}。
\item 交变电流的频率在数值上等于电流在每秒内周期性变化的次数。
\item 我国电网以交流供电,频率为 \blue{50Hz}。
}