\section{磁现象}

\Itemize{
\item 能够吸引\blue{铁}、\blue{钴}、\blue{镍}等物质的性质叫做\Keyword{磁性}{magnestion}。
\item 具有磁性的物体叫做\Keyword{磁体}{magnet}。
\item 磁性最强的两个部位叫做\Keyword{磁极}{magnetic pole}。
\item 能够自由转动的磁体,静止时由指向北方的磁极叫做\Keyword{北极}{north pole}或 \blue{N 极},指向南方的磁极叫做\Keyword{南极}{south pole}或 \blue{S 极}。
\item 磁极间相互作用的规律,即\blue{同名磁极相互排斥},\blue{异名磁极相互吸引}。
\item 原本没有磁性的物体在磁体或电流的作用下获得磁性的过程叫做\Keyword{磁化}{magnestization}。
\item 能够被磁化的物质统称为\blue{磁性材料}。
\item 磁性材料分为硬磁性材料和软磁性材料。
\item 被磁化后能够长期保持磁性的材料叫做\blue{硬磁性材料}(永磁体)。被磁化后不能长期保持磁性的材料叫做\blue{软磁性材料}。
}