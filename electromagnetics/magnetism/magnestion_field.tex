\section{磁场}

\Itemize{
\item \Keyword{磁场}{magnestion field}的\blue{基本性质},即磁场对放入其中的磁体有力的作用。
\item 磁场中的不同位置磁场的强弱和方向不同。
\item \blue{规定}小磁针静止时 \blue{N 极所指的方向}为这点磁场的方向。
\item \Keyword{磁感线}{magnetic induction line}\blue{不存在},只是为了方便形象地描述磁场。
\item 磁体外部的磁感线都是从磁体的 \blue{N 极出发回到 S 极}。
\item \blue{磁感线疏密表示磁场强弱}。磁感线稀疏的地方磁场弱,磁感线密集的地方磁场强。
\item 磁感线上某点的\blue{切线方向},既是放在该处的小磁针 \blue{N 极的受力方向},也是该点的\blue{磁场方向}。
\item 地球周围空间存在的磁场叫做\blue{地磁场}。
\item 地磁的 N 极在地理的南极附近,地磁的 S 极在地理的北极附近。
\item 地磁场的两级和地理的两级不重合。
\item 地磁场的磁感线分布跟条形磁体的磁场相似。
\item 磁针所指南北方向偏离地理南北方向的角度叫做\blue{磁偏角}。
}