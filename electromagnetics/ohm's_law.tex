\section{欧姆定律}

\subsection{电流与电压、电阻的关系}
\begin{itemize}
\item 在\blue{电阻一定}时,通过导体的电流与导体两端的电压成正比.
\item 在\blue{电压一定}时,通过导体的电流与导体的电阻成正比.
\item \blue{欧姆定律}(Ohm's law),即导体中的电流,跟导体两端的电压成正比,跟导体的电阻成反比. 有:
\mathline{I=\frac UR}
\newline 欧姆定律对金属、电解液适用,对半导体、电离气体不适用.
\end{itemize}

\subsection{电阻的测量}
\begin{itemize}
\item 伏安法测电阻,即利用 $R=\frac UI$ 测量电阻.
\item 小灯泡是非线性元件,其伏安特性曲线如图.
\begin{figure}[H]
	\centering
	\begin{tikzpicture}
	\begin{axis}[
		axis lines = middle,
		xmin = 0, xmax = 10,
		ymin = 0, ymax = 10,
		smooth, thick,
		xlabel = {$U/\text{V}$}, ylabel = {$I/\text{A}$},
		xlabel style = {anchor = north},
		ylabel style = {anchor = east},
		xtick = \empty,	ytick = \empty,
		samples = 200,
	]
		\addplot+[no marks, domain = 0 : 10]{log10(x + 1) / log10(1.35)};
	\end{axis}
	\node at (0, 0) [below = 5pt, left = -2pt] {$O$};
	\end{tikzpicture}
	\titlename{小灯泡的伏安特性曲线}
\end{figure}
\end{itemize}

\subsection{串、并联电路中的分压、分流规律}
\begin{itemize}
\item 串联分压,即 \blue{$\bm{U_1:U_2:\cdots:U_n=R_1:R_2:\cdots:R_n}$}.
\item 并联分流,即 \blue{$\bm{I_1:I_2:\cdots:I_n=\frac1{R_1}:\frac1{R_2}:\cdots:\frac1{R_n}}$}.
\end{itemize}