\section{电路}

\subsection{简单电路}
\begin{itemize}
\item \blue{电路}(electric circuit),即用导线将用电器、电源、开关连接起来.
\item \blue{电源}(power supply),即提供电能的装置,如电池、发电机.
\item \blue{用电器},即消耗电能的装置,如灯泡、电动机.
\item \blue{开关},即控制电路通断的装置,如单刀单掷开关、单刀双掷开关.
\item \blue{导线}通常由绝缘外皮和金属内芯(铜或铝)组成.
\item 处处连通的电路叫做\blue{通路}(\blue{闭合电路}). 某处断开的电路叫做\blue{断路}(\blue{开路}).
\item \blue{直接}用导线将电源的正、负极连接起来的电路叫做\blue{短路}.
\item 闭合电路中,用电器两端被导线直接连通叫做用电器被\blue{短接}.
\item 用符号表示电路连接的图叫做\blue{电路图}.
\item \blue{串联}(series connection)和\blue{并联}(parallel connection).
\item \blue{串联电路}和\blue{并联电路}.
\item 串联电路中各用电器相互影响,并联电路各用电器互不影响.
\end{itemize}

\subsection{电源}
\begin{itemize}
\item 能把电子从 A 搬运到 B 的装置 P 就是\blue{电源}(power source). A 和 B 是电源的两个\blue{电极}.
\end{itemize}

\subsection{电流及其测量}
\begin{itemize}
\item 电荷的\blue{定向移动}形成电流(electric current).
\item 电路只有闭合时,电路中才有电流.
\item 规定\blue{正电荷}定向移动的方向为电流的方向.
\item 电子向某一方向定向移动等效于正电荷向相反方向定向移动.
\item 电路闭合时,\blue{电源外部}电流的方向是从电源正极经过用电器流向电源负极.
\item 电流强度是表示\blue{电流强弱程度}的物理量.
\item 单位时间内通过导体横截面的电荷量叫做\blue{电流强度},简称\blue{电流}(electric current),用 \blue{$\bm I$} 表示. 用 $q$ 表示在时间 $t$ 内通过导体横截面的电荷量,则有:
\mathline{I=\frac qt}
\item 在国际单位制中,电流的单位是\blue{安培}(ampere),简称\blue{安}. 符号是 \blue{$\bf A$}. \blue{$\bf 1A=1C/s$}. 常用单位还有\blue{毫安}(\blue{$\bf mA$})和\blue{微安}(\blue{$\bf\textmu A$}),它们与安培的关系是 \blue{$\bf1mA=10^{-3}A$},\blue{$\bf1\textmu A=10^{-6}A$}.
\item 导体的横截面积为 $S$,\blue{自由电子数密度}(单位体积内的自由电子数)为 $n$,自由电子定向移动的平均速率为 $v$,电子的电荷量为 $e$,则:
$$
I=neSv
$$
\item 测量电路中电流大小的仪表叫做\blue{电流表},符号是\ammeter.
\end{itemize}

\subsection{电压及其测量}
\begin{itemize}
\item \blue{电压}(voltage)用 \blue{$\bm U$} 表示. 单位是\blue{伏特}(volt),简称\blue{伏}. 符号是 \blue{$\bf V$}. 常用单位还有\blue{千伏}(\blue{$\bf kV$})和\blue{毫伏}(\blue{$\bf mV$}),它们与伏特的关系是 \blue{$\bf1kV=10^3V$},\blue{$\bf1mV=10^{-3}V$},\blue{$\bf1\textmu V=10^{-6}V$}.
\item 干电池的电压为 1.5V,铅蓄电池的电压为 2V.
\item 测量电路中两点间电压大小的仪表叫做\blue{电压表},符号是\voltmeter.
\end{itemize}

\subsection{串、并联电路中电流、电压的规律}
\begin{itemize}
\item 在串联电路中,电流处处相等. 在并联电路中,干路电流等于各支路电流之和.
\item 在串联电路中,总电压等于各用电器两端电压之和. 在并联电路中,各支路两端电压相等,等于总电压.
\item 串联电池组两端电压等于每节电池两端电压之和. 并联电池组两端电压等于每节电池两端电压.
\end{itemize}

\subsection{电阻}
\begin{itemize}
\item 容易导电的物体叫做\blue{导体}(conductor). 不容易导电的物体叫做\blue{绝缘体}(insulator).
\item 导电性能介于导体和绝缘体之间的物体叫做\blue{半导体}(semiconductor).
\item 导体中有大量的能够自由移动的电荷(\blue{自由电荷}),而绝缘体很少.
\item 在外界\blue{温度}、\blue{压力}、\blue{光照}等条件发生改变或掺入杂质时,绝缘体有可能变成导体.
\item \blue{电阻}(resistance)是表示导体对电流阻碍作用大小的物理量,用 \blue{$\bm R$} 表示. 单位是\blue{欧姆},简称\blue{欧},符号是 \blue{$\bm\Omega$}. 常用单位还有千欧(\blue{${\bf k}\bm\Omega$})、兆欧(\blue{${\bf M}\bm\Omega$}),换算关系为 \blue{$\bf1k\bm\Omega=10^3\bm\Omega$},\blue{$\bf1M\bm\Omega=10^6\bm\Omega$}.
\item 具有一定电阻值的元件叫做\blue{电阻器},也叫做\blue{定值电阻},简称\blue{电阻},符号是\resistance. 
\item \blue{电流表的电阻很小},\blue{电压表的电阻很大}.
\item 导体的电阻与导体的材料、长度、横截面积和温度有关.
\item \blue{$\bm{R=\dfrac UI}$}.
\item \blue{电阻定律},即同种材料的导体,其电阻 \blue{$\bm R$} 与它的长度 \blue{$\bm l$} 成正比,与它的横截面积 \blue{$\bm S$} 成反比. 导体电阻还与\blue{材料}有关. 有:
\mathline{R=\rho\frac lS}
\newline \blue{$\bm\rho$} 叫做材料的\blue{电阻率}(resistivity).
\item \blue{金属的电阻率随温度的升高而增大}.
\item 当温度降低到某一温度时,物质的\blue{电阻变为零},这种现象叫做\blue{超导现象}. 发生超导现象的物质叫做\blue{超导体}(superconductor),物质出现超导现象的温度叫做\blue{临界温度}或\blue{转变温度}.
\item 用横坐标表示电压 $U$,纵坐标表示电流 $I$. 画出的 $I-U$ 图像叫做导体的\blue{伏安特性曲线}.
\item 电流与电压成正比的电学元件叫做\blue{线性元件}. 电流与电压不成正比的电学元件叫做\blue{非线性元件}.
\item 能改变接入电路中电阻大小的元件叫做\blue{变阻器}. 其作用包括保护电路、改变电流、控制电压.
\item \blue{滑动变阻器}的符号是\slidingrheostat.
\end{itemize}