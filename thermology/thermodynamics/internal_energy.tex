\section{内能}

\subsection{内能}
\Itemize{
\item 分子由于\blue{热运动}而具有的能量叫做\blue{分子动能}。
\item 系统中所有分子的动能的平均值叫做\blue{分子热运动的平均动能}。
\item 物体温度升高时,分子热运动的平均动能增加。
\item \blue{温度}是\blue{分子热运动的平均动能}的标志。
\item 单原子分子的平均动能 $\overline{E_k}=\frac32kT$,即 $\overline{E_k}\propto T$。其中 $k$ 是玻尔兹曼常数。
\item 分子之间由于存在\blue{相互作用力}而具有的能叫做\blue{分子势能}。
\item 分子势能 $E_p$ 与分子间的距离 $r$ 有关。即:
\newline 当 $r=r_0$ 时,分子间的作用力 $F$ 为 0,\blue{分子势能最小}。
\newline 当 $r>r_0$ 时,分子间的作用力 $F$ 表现为引力,分子势能减小。
\newline 当 $r<r_0$ 时,分子间的作用力 $F$ 表现为斥力,分子势能增大。
\item 分子势能的大小由\blue{分子间的相对位置}决定。如果选定分子间距离 $r$ 为无穷远时的分子势能 $E_p$ 为 0,则分子势能 $E_p$ 随分子间距离变化的情况如\FigureRef{分子势能与分子间的距离的关系}所示。
\begin{figure}[H]
	\centering
	\begin{tikzpicture}
	\begin{axis}[
		axis lines = middle,
		xmin = 0, xmax = 4,
		ymin = -2, ymax = 2,
		smooth, thick,
		xlabel = {$r$}, ylabel = {$E_p$},
		xlabel style = {anchor = north},
		ylabel style = {anchor = east},
		xtick = {0.88205}, xticklabels = {$r_0$}, xtick style = {anchor = north east},
		ytick = \empty,
		samples = 200,
	]
		\addplot+[no marks, domain = 0.25 : 3.75]{9 / ((x + 0.85) ^ 6) - 3 / ((x + 0.85) ^ 2)};
	\end{axis}
	\node at (0, 2) [below = 5pt, left = -1pt] {$O$};
	\end{tikzpicture}
	\Title{分子势能与分子间的距离的关系}
\end{figure}
\item 分子势能与物体体积有关。
\item 物体中所有分子的\blue{分子动能与分子势能的总和},叫做物体的\Keyword{内能}{internal energy}。任何物体都具有内能。内能的单位是\blue{焦耳($\bf J$)}。
\item 物体的内能与\blue{温度}和\blue{体积}有关。
}

\subsection{比热容}
\Itemize{
\item 内能由\blue{高温}物体转移到\blue{低温}物体的过程叫做\blue{热传递}。
\item 热传递的基本方式包括传导、对流和辐射。
\item 在热传递过程中,传递能量的多少叫做\Keyword{热量}{quantity of heat}。用符号 \blue{$\bm Q$} 表示。单位是\blue{焦耳}。
\item 物体吸收热量是内能增加,放出热量时内能减少。\blue{热量是物体内能改变的量度}。
\item 一定质量的某种物体,在温度升高(或降低)时吸收(或放出)的热量与它的质量和升高(或降低)的温度乘积之比,叫做这种物质的\Keyword{比热容}{specific heat capacity}。用符号 \blue{$\bm c$} 表示。单位是\blue{焦每千克摄氏度($\bf J/(kg\cdot\textcelsius)$)}。有:
\Math{c=\frac{\Delta Q}{m\Delta t}}
\item 比热容是反映\blue{物质自身性质}的物理量。
\item 不同的物质,比热容一般不同。
\item 水的比热容为 \blue{$\bf4.2\times10^3J/(kg\cdot\textcelsius)$}。
\item 热量的计算有 \blue{$\bm{\Delta Q=cm\Delta t}$}。
\item \blue{热平衡方程},即 \blue{$\bm{\Delta Q_{\tiny\mbox{吸}}=\Delta Q_{\tiny\mbox{放}}}$}。
}