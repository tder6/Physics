\section{热力学定律}

\vspace{10pt}
\begin{itemize}
\item 热学包括\blue{热力学}和\blue{统计物理}。
\item 热力学的研究对象叫做\blue{热力学系统}(thermodynamic system),简称\blue{系统}。热力学系统是由大量分子组成的。
\item 系统之外与系统发生相互作用的其他物体统称为\blue{外界}。
\end{itemize}

\subsection{系统内能的改变}
\begin{itemize}
\item 改变系统内能的两种方式是\blue{热传递}和\blue{做功}。
\item 在热传递过程中,系统吸收热量内能增加,放出热量内能减少。
\item 热量是\blue{热传递过程中}系统内能变化的量度。
\item 系统与外界没有热传递的过程叫做\blue{绝热过程}(adiabatic process)。
\item 在绝热过程中,外界对系统做功,系统内能增加;系统对外界做功,系统内能减少。
\item 做功是\blue{绝热过程中}系统内能变化的量度。
\item 焦耳的实验表明\blue{热传递和做功对改变系统的内能是等效的}。
\end{itemize}

\subsection{热力学第一定律}
\begin{itemize}
\item \blue{热力学第一定律}(first law of thermodynamics),即热力学系统内能 $U$ 的变化量等于系统从外界吸收的热量(系统向外界放出的热量)与外界对系统做的功(系统对外界做的功)之和。有:
\mathline{\Delta U=Q+W}
系统对外界吸热,$Q$ 为正值;系统对外界放热,$Q$ 取负值;外界对系统做功,$W$ 取正值;系统对外界做功,$W$ 取负值。
\end{itemize}

\subsection{能量守恒定律}
\begin{itemize}
\item 不同形式的能量可以在一定条件下相互转化。
\item \blue{能量守恒定律}(law of conservation of energy),即能量既不会凭空产生,也不会凭空消失,它只能从一种形式\blue{转化}为其他形式,或者从一个物体\blue{转移}到其他物体,在转化或转移的过程中,能量的总量保持不变。
\item 能量守恒定律是自然界最普遍、最重要的基本定律之一。
\item 能量守恒定律的本质是\blue{时间平移对称性}。
\item \blue{永动机},即不需要动力就能源源不断地对外做功的机器,分为第一类永动机和第二类永动机。
\item 能量守恒定律的另一种表述为\blue{第一类永动机不可能制成}。
\end{itemize}

\subsection{热力学第二定律}
\begin{itemize}
\item 一切与热现象有关的宏观自然过程都是\blue{不可逆}的。
\item \blue{热力学第二定律}(second law of thermodynamics)。
\item \blue{克劳修斯表述},即热量不能\blue{自发}地从低温物体传到高温物体。自发是指不需要任何第三者的介入,不会对任何第三者产生任何影响。自发的方向是从高温物体指向低温物体。
\item 克劳修斯表述阐述了\blue{传热的方向性}。
\item \blue{开尔文表述},即\blue{不可能从单一热库吸收热量},使之完全变成功,而不产生其他影响。不可能从单一热库吸热,而且一定会向另一个热库放热。
\item 开尔文表述阐述了\blue{机械能与内能转化的方向性}。
\item \blue{热力学第二定理的另一种表述为}\blue{第二类永动机不可能制成}。
\item 热力学第二定律的克劳修斯表述和开尔文表述是\blue{等价的}。
\item \blue{能量耗散},即不同形式的能量最终都转化为\blue{内能}并分散在环境中的过程。
\end{itemize}
